% vim: tw=80 cc=80
\documentclass[10pt,a4paper]{article}
\usepackage[cm]{fullpage}
\usepackage{hyperref}
\usepackage{caption}
\usepackage{listings}
\usepackage{enumitem}
\usepackage{xcolor}
\usepackage{float}
\usepackage{tikz}
\usetikzlibrary{arrows,automata}

\definecolor{codegreen}{rgb}{0,0.6,0}
\definecolor{codegray}{rgb}{0.5,0.5,0.5}
\definecolor{codepurple}{rgb}{0.58,0,0.82}
\definecolor{backcolour}{rgb}{0.95,0.95,0.95}

\lstdefinestyle{mystyle}{
    backgroundcolor=\color{backcolour},
    commentstyle=\color{codegreen},
    keywordstyle=\color{magenta},
    numberstyle=\tiny\color{codegray},
    stringstyle=\color{codepurple},
    basicstyle=\ttfamily\footnotesize,
    breakatwhitespace=false,
    breaklines=true,
    captionpos=b,
    keepspaces=true,
    numbers=left,
    numbersep=5pt,
    showspaces=false,
    showstringspaces=false,
    showtabs=false,
    tabsize=2,
    frame=single
}

\lstset{style=mystyle}

\setlength{\parindent}{0cm}
\setlength{\parskip}{8pt}

\begin{document}

\section{Z80-Reto-cpmlib}

This guide is intended provide some commentary that can be read along with the
source code to showcase the various components of the library and how they can
be used in your own applications.

You should read the comments in the header files.  They are updated during
development and will be the most accurate.

\subsection{What Is Z80-Retro-cpmlib?}

Z80-Retro-cpmlib is a C library that you can link to in your applications for
use with the
\href{https://github.com/etchedpixels/fuzix-compiler-kit.git}{FUZIX-Compiler-Kit}.
The library is specifically targeted at the
\href{https://github.com/z80-retro}{Z80-Retro!} Single Board Computer by John
Winans.

Some standard libraries are implimented where they can be easily backed by the
CP/M 2.2 BDOS function calls.  Additionally the library includes functions for
working with the TMS9918 and Atari style joystick ports on the VDP daughter
board designed for the Z80-Retro!.

For instructions on how to install the compiler and this library, see the
\url{./BUILD.md} documentation.

\subsection{Headers}

The headers are all located in the projects "include" directory.  You just need
to \texttt{\#include} the ones you need and make sure to link to the
\texttt{cpmlib.a} library.

As the code is split out into multiple translation units, your resulting binary
should include almost no wasted code.

\subsection{Usage}

You can use the by including the headers you need, calling the functions in
your code and finally compilling and linking. See: "Listing: \ref{lst:makefile}
- Example Makefile" on page \pageref{lst:makefile}

The process is something like this:

\begin{description}
  \item[Compile] fcc -O2 -mz80 -Iinclude -I /opt/fcc/lib/z80/include -c -o
    main.o main.c
  \item[Link] ldz80 -b C0x100 -o main.bin crt0.o main.o libcpm.a
  \item[Truncate] dd if=main.bin of=main.com skip=1 bs=256
\end{description}

The linker documentation is very minimal because it's a very minimal linker.
The \texttt{`-b`} switch tells the linker to output a binary file without
relocatable code.  The -C0x100 tells the linker to begin the \texttt{`code`}
segment at 0x100 which is the beginning of the TPA for CP/M.

Because the linker always starts filling code from 0x0000 we need to remove the
first 256 bytes using the \texttt{`dd`} command.
\pagebreak

\begin{lstlisting}[language=make, caption=Example Makefile]
TOP=.
CC=/opt/fcc/bin/fcc
AS=/opt/fcc/bin/asz80
LD=/opt/fcc/bin/ldz80

CFLAGS=-O2 -mz80 -I $(TOP)/../include -I /opt/fcc/lib/z80/include
LDFLAGS=-b -C0x100
LIBS=\
		 $(TOP)/../cpmlib.a \
		 /opt/fcc/lib/z80/libz80.a \
		 /opt/fcc/lib/z80/libc.a

CRT=$(TOP)/../crt0.o

all: clean malloc.com fileio.com testtms.com copy

malloc.bin: malloc.o
	$(LD) $(LDFLAGS) -o $@ $(CRT) $^ $(LIBS)

malloc.com: malloc.bin
	dd if=$^ of=$@ skip=1 bs=256

fileio.bin: fileio.o
	$(LD) $(LDFLAGS) -o $@ $(CRT) $^ $(LIBS)

fileio.com: fileio.bin
	dd if=$^ of=$@ skip=1 bs=256

testtms.bin: testtms.o
	$(LD) $(LDFLAGS) -o $@ $(CRT) $^ $(LIBS)

testtms.com: testtms.bin
	dd if=$^ of=$@ skip=1 bs=256


clean:
	rm -fv malloc.bin malloc.com fileio.bin fileio.com
	find . -name "*.o" -exec rm -fv {} \;
\end{lstlisting}
\label{lst:makefile}

This example does not show the actual FCC commands explicitly.  Make is
automating that step for us.
\pagebreak

\section{C Runtime}

\break

\section{string}

\pagebreak

\section{TMS99xx}

\pagebreak

\section{CP/M}
The CP/M header provides C wrappers for almost all the CP/M BDOS function calls.

For example, if you want to check for keyboard input without blocking, you can
call the \texttt{cpm\_rawio()} function which returns the ascii char or 0.
Unlike \texttt{cpm\_conin()}, this function does not emit the typed character to
the terminal.

You can find an example of how to use the fileio functions in the "test"
folder.  The "fcntl.h" header file contains lots of additional information.

\pagebreak

\section{malloc, free}
There are two parts to the malloc / free implimentation in this library.  The
first is the implimentation of the "sbrk()" system call and the second is the
malloc and free functions themselves.

The sbrk() function was copied and adapted from the HiTech C compiler project.

The malloc and free functions are transcribed directly from "The C Programming
Language - Second Edition" By Biran W. Kernignhan and Dennis M. Ritchie.

There are is no defragmentation or garbage collection.  Malloc doesn't usually
make all that much sense in embedded environments.


\pagebreak

\end{document}
